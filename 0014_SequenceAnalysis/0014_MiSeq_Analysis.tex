% Options for packages loaded elsewhere
\PassOptionsToPackage{unicode}{hyperref}
\PassOptionsToPackage{hyphens}{url}
%
\documentclass[
]{article}
\usepackage{amsmath,amssymb}
\usepackage{lmodern}
\usepackage{iftex}
\ifPDFTeX
  \usepackage[T1]{fontenc}
  \usepackage[utf8]{inputenc}
  \usepackage{textcomp} % provide euro and other symbols
\else % if luatex or xetex
  \usepackage{unicode-math}
  \defaultfontfeatures{Scale=MatchLowercase}
  \defaultfontfeatures[\rmfamily]{Ligatures=TeX,Scale=1}
\fi
% Use upquote if available, for straight quotes in verbatim environments
\IfFileExists{upquote.sty}{\usepackage{upquote}}{}
\IfFileExists{microtype.sty}{% use microtype if available
  \usepackage[]{microtype}
  \UseMicrotypeSet[protrusion]{basicmath} % disable protrusion for tt fonts
}{}
\makeatletter
\@ifundefined{KOMAClassName}{% if non-KOMA class
  \IfFileExists{parskip.sty}{%
    \usepackage{parskip}
  }{% else
    \setlength{\parindent}{0pt}
    \setlength{\parskip}{6pt plus 2pt minus 1pt}}
}{% if KOMA class
  \KOMAoptions{parskip=half}}
\makeatother
\usepackage{xcolor}
\usepackage[margin=1in]{geometry}
\usepackage{graphicx}
\makeatletter
\def\maxwidth{\ifdim\Gin@nat@width>\linewidth\linewidth\else\Gin@nat@width\fi}
\def\maxheight{\ifdim\Gin@nat@height>\textheight\textheight\else\Gin@nat@height\fi}
\makeatother
% Scale images if necessary, so that they will not overflow the page
% margins by default, and it is still possible to overwrite the defaults
% using explicit options in \includegraphics[width, height, ...]{}
\setkeys{Gin}{width=\maxwidth,height=\maxheight,keepaspectratio}
% Set default figure placement to htbp
\makeatletter
\def\fps@figure{htbp}
\makeatother
\setlength{\emergencystretch}{3em} % prevent overfull lines
\providecommand{\tightlist}{%
  \setlength{\itemsep}{0pt}\setlength{\parskip}{0pt}}
\setcounter{secnumdepth}{-\maxdimen} % remove section numbering
\ifLuaTeX
  \usepackage{selnolig}  % disable illegal ligatures
\fi
\IfFileExists{bookmark.sty}{\usepackage{bookmark}}{\usepackage{hyperref}}
\IfFileExists{xurl.sty}{\usepackage{xurl}}{} % add URL line breaks if available
\urlstyle{same} % disable monospaced font for URLs
\hypersetup{
  pdftitle={0014\_MiSeq\_Analysis},
  pdfauthor={Walker Symonds-Orr},
  hidelinks,
  pdfcreator={LaTeX via pandoc}}

\title{0014\_MiSeq\_Analysis}
\author{Walker Symonds-Orr}
\date{2023-08-30}

\begin{document}
\maketitle

First, access the cluster. Make sure your PIV card is inserted.

At the terminal, type:

\begin{verbatim}
ssh orrwe@ai-submit1.niaid.nih.gov
\end{verbatim}

Before using the cluster, you need to create your own node, so you are
not using common resources on the cluster. You will get automated (and
even) emails from IT if you fail to do this!

\begin{verbatim}
qrsh
\end{verbatim}

qrsh has other arguments as well, which can be used to control the size
of the node you get, for example.

Before going forward: remember that many of the directories on the
cluster are shared, and that it's very important to treat it with
respect and not do anything of which consequence you are uncertain.

\hypertarget{demultiplexing}{%
\subsection{Demultiplexing}\label{demultiplexing}}

The first step in sequence analysis is de-multiplexing, which means
taking bcl binary files from the sequencer and converting them to fastq
files while at the same time identifying which adapter index belongs to
which sample.

Identify the directory that contains your sequencing data. First, you
will want to connect to LOCUS where our data is stored. In finder, press
⌘K (Go \textgreater{} Connect to Server). Then type

\begin{verbatim}
smb://locusdata.niaid.nih.gov/
\end{verbatim}

and select LVD\_QVE.

Within this, select the folder for Sequencing Data and navigate to your
sequencing run.

create a new directory to house your demultiplexed sequences called
``demux'' or something to that effect.

blc2fastq is a ``module'', which is like an application you can use on
the cluster. Before using it, you must load it. To see available
modules, type

\begin{verbatim}
module avail #followed by a string representing a search of the modules
\end{verbatim}

If you don't supply an argument afterwards, this command will produce a
long list of modules. Type bcl and you will get a list of bcl2fastq
modules. Choose the most current iteration by typing:

\begin{verbatim}
module load bcl2fastq2/2.20
\end{verbatim}

Remember that you can use tab-complete!

bcl2fastq module is now loaded on your node. You can type

\begin{verbatim}
bcl2fastq help
\end{verbatim}

to get a sense of what arguments this function takes.

When you are ready to demultiplex the dna, you will type

\begin{verbatim}
bcl2fastq -i "input_directory" -R "run_directory" -o "output_directory"
\end{verbatim}

The arguments in quotes will be replaced with the specific directories
at hand.

The input directory {[}-i{]} will be inside the directory with the
sequencing run data created by the MiSeq. This will be under
./Data/Intensities/BaseCalls. You can copy the folder as pathname or use
tab-complete. If you copy as pathname make sure to change the beginning
of the address from /Volumes/LVD\_QVE to /hpcdata/lvd\_qve.

An example from a recent sequencing run I did is:

\begin{verbatim}
/hpcdata/lvd_qve/Sequencing_Data/QVEU_Seq_0087_Miseq_Capsid_Deletion_Library_EV71_WalkerOrr/QVEU_Seq0087_WO_WB_04_Capsiddel_Replicationdel_DIMPLE/230820_M02211_0114_000000000-KD2BY/Data/Intensities/BaseCalls
\end{verbatim}

The Run directory {[}-R{]} (note the capital R) will be the root
directory from the sequencing run.

For the output directory {[}-o{]}, use the demux folder you just
created.

\end{document}
